% SPF Report:
\documentclass[english]{SPFReport}

%\usepackage{spfFigures}
\usepackage{longtable}
\usepackage{amsfonts}
\usepackage{color,xcolor}
\usepackage{listings}
\usepackage{footnote}
\usepackage{longtable}
\usepackage{siunitx}
\usepackage[obeyspaces]{url}
\usepackage[sfdefault,condensed]{roboto}
\sisetup{detect-all=true,text-rm=\robotocondensed}

\newcommand\Tstrut{\rule{0pt}{2.6ex}}         % = `top' strut
\newcommand\Bstrut{\rule[-0.9ex]{0pt}{0pt}}   % = `bottom' strut

\renewcommand{\thefootnote}{\thempfootnote}

% Environment Variables:
\newcommand{\ud}{\mathrm{d}}
\newcommand{\HRule}{\rule{\linewidth}{0.5mm}}
%\renewcommand{\thefootnote}{\fnsymbol{footnote}} 	


\definecolor{dkgreen}{rgb}{0,0.6,0}
\definecolor{gray}{rgb}{0.5,0.5,0.5}
\definecolor{mauve}{rgb}{0.58,0,0.82}
 
%\lstset{ %
%  language=[90]Fortran,           % the language of the code
%  basicstyle=\footnotesize\ttfamily,       % the size of the fonts that are used for the code
%  numbers=left,                   % where to put the line-numbers
%  numberstyle=\tiny\color{gray},  % the style that is used for the line-numbers
%  stepnumber=1000,                   % the step between two line-numbers.
%  numbersep=1pt,                  % how far the line-numbers are from the code
%  backgroundcolor=\color{white},  % choose the background color. You must add \usepackage{color}
%  showspaces=false,               % show spaces adding particular underscores
%  showstringspaces=false,         % underline spaces within strings
%  showtabs=false,                 % show tabs within strings adding particular underscores
%  extendedchars=true,            % allow special characters
%  rulecolor=\color{black},        % if not set, the frame-color may be changed on line-breaks
%  tabsize=1,                      % sets default tabsize to 2 spaces
%  captionpos=b,                   % sets the caption-position to bottom
%  breaklines=true,                % sets automatic line breaking
%  breakatwhitespace=false,        % sets if automatic breaks should only happen at whitespace
%  keywordstyle=\color{blue},      % keyword style
%  commentstyle=\color{dkgreen},   % comment style
%  stringstyle=\color{mauve},      % string literal style
%  morekeywords={double,precision}            % if you want to add more keywords to the set
%}

\reportName{Config files }
\reportSubName{Explanation of confing files for running and processing TRNSYS files useing PyTrnsys}

\reportDate{14$^{st}$ October of 2019} % or writte the date manually 
\author{Dr. Daniel Carbonell}
\address{Dani.Carbonell@spf.ch}

\abstract{ 
This report explains the main functionalities included in the config files for the running and processing of TRNSYS results.
}

\begin{document} 


\section{Introduction}

The idea of the config files is to include general functionality without having to type python code for the user.
Thus, this config file will grow as long as users believe some functionality is used so often that its worth to implemente it within the config file. All the functionality not included in the config file will need to be implemented as python code and thus will be a bit limited to those knowing how to program in python. 

There are several types of variables with scape separator:
\begin{itemize}
\item \textbf{bool} :with True/False as possible value
\item \textbf{int} : with any interger as possible value
\item \textbf{string} : with ``any string'' as possible value
\item \textbf{deck} : with any float/int as possible value. If a string is needed then the format deck name string anystring should be used
\item \textbf{STRING\$}: ddck file path. STRING\$ needs to define one path of the ddck folder. 
\item \textbf{variation} : used to do parametric variations of chaginfg deck variables. For parallel simulations
\item \textbf{changeDdckFile} : used to do changes on the ddck files loaded in the config, e.g. weather data
\item \textbf{fit} : ask Mattia
\item \textbf{case} : ask Mattia
\item \textbf{fitobs} : ask Mattia



\end{itemize}
\section{Revison history}


\begin{tabular}{| l |  l | l  | p{80mm} | }\hline

    \textbf{Date} & \textbf{Version} & \textbf{Author}&\textbf{Changes}  \\\hline
\Tstrut\Bstrut     Oct 14, 2019 & v1.0 & D. Carbonell&First version released \\\hline
  \end{tabular}

\section{Config file for running TRNSYS cases}

The separator used is a blank space.
%\begin{minipage}

\begin{tabular}{| p{25mm} |  p{35mm} | p{90mm}  |}
\hline
\textbf{Type} & \textbf{Name} & \textbf{Description} \\
\hline
 %\multicolumn{4}{c}{\small Booleans}  \\ \hline \hline
 bool & ignoreOnlinePlotter &  It comments all online plotters  \\
 bool & runCases & if set to False it created all cases but do not execute TRNSYS  \\
% bool & runFromCases & True/False. 
 %bool & runFromFolder & True/False.  \\
 %bool & rerunCases & to check  \\
 bool & parseFileCreated &  to check  \\
 bool & combineAllCases & True : all possible combinations of the variation are used; False : variations are executed case to case, so all of them need to be of the same size\\
% bool & copyBuildingData &  of no use anymore?   \\
 % \multicolumn{4}{c}{\small Integers} \\
 int & reduceCpu &  uses the value to reduce the number of CPU. \\&&Suggested 1 or 2 if you would like to work when parallel simulations are running  \\
 % \multicolumn{4}{c}{\small Strings} \\
string & runType & \textbf{``runFromCases''} : runs the cases defined in a file and path defined by strings pathWithCasesToRun and fileWithCasesToRun. \\
&& \textbf{``runFromFolder''} : It executes all the cases found in a specific folder defined by string pathFolderToRun. This used to execute failed cases that were moved to a specific folder. \\
&& \textbf{``runFromConfig''} : it builts and executes the cases as defined in the config file \\
string & pathWithCasesToRun & `'path'' of file with cases to run. Only active if runFromCases=True \\
string & fileWithCasesToRun & `fileName'' added to the pathWithCasesToRun defined. Only active if runFromCases=True \\
string & pathFolderToRun & `'path'' of the folder with all cases to run. Only active if runFromFolder=True \\
 string & trnsysExePath & including the exe file, e.g. \textit{"C:/Trnsys17/Exe/TRNExe.exe"} \\
 string & addResultsFolder & False to deactivate. If defined as a ``stringPath'' it creates a folder on the main path and creates all simulations folders inside this folder path\\
  string & STRING\$ & the symbol \$ denotes a path used for find ddck files. The STRING\$ defined can be used as path to look for the ddck files.\\
 string & scaling & False or "toDemand" or ``toNormalDemand'' (custom for TRI-HP) \\
string &pathRef & path to look for th reference case if scaling !=False (custom for TRI-HP)  \\
string &nameRef & file to look for th reference case if scaling !=False  (custom for TRI-HP) \\

% string & commonTrnsysFolder & needed ?? \\
  % \multicolumn{4}{c}{\small Variation} \\
  variation & list & It builds dck files using all variation given.  By defaul all permutations are used. Format: variation nameFolderResults nameVariableDDck listOfvalues.\\
  % \multicolumn{4}{c}{\small Deck} \\
    deck & any ddck variable & It changes the value (int/float) of the variable. Format deck name value(int/float) or deck name string anystring (used for equations) \\
    
changeDDckFile & list of names &  nameToChange changedName1 changedName2 etc \\
addDDckFolder & liost of names &  folderForChangeName1 folderForChangeName2 \\

   % \multicolumn{4}{c}{\small List of Ddck files to built the dck} \\
   STRING\$ & ddckNamePath & The ddck file path will be added to the STRING\$ path \\   
% \multicolumn{4}{c}{\small -- continued on next page --} \\
\hline
\end{tabular}


\section{Example}
\# used to filter lines and comments \\
\textcolor{blue}{bool} ignoreOnlinePlotter  True  \\
\textcolor{blue}{int} reduceCpu  1  \\
\textcolor{blue}{bool} parseFileCreated True  \\
\textcolor{blue}{bool} runCases False  \\

\textcolor{blue}{string} trnsysExePath \path{"C:\Trnsys17\Exe\TRNExe.exe"} \\
\textcolor{blue}{string} addResultsFolder "DebugHydCool"  \\

\path{######################} \\
\# DDeck paths definition \\
\path{######################} \\
\textcolor{blue}{string} HOME\$ \path{"C:\Daten\spfPackages\GIT\TriHpTrnsysDDeck"} \\
\textcolor{blue}{string} SPF\$  \path{"C:\Daten\spfPackages\GIT\spfTrnsysFiles"} \\


\textcolor{blue}{string} scaling False \#"toDemand" \\
\textcolor{blue}{string} pathRef  \path{"C:\Daten\OngoingProject\\BigIce\Simulations\BICE-HydD_circ-8Cities-Ref"} \\
\textcolor{blue}{string} nameRef  \path{"BICE-HydD_circ-Ac136.4-Vice27.3-MFH"} \\

\path{######################} \\
\# Type of Running \\
\path{######################} \\
\textcolor{blue}{string} runType "runFromConfig" \#"runFromCases","runFromFolder" \\
\path{############} if runType = "runFromCases" \\
\textcolor{blue}{string} fileWithCasesToRun "TestCases.txt" \\
\textcolor{blue}{string} pathWithCasesToRun \path{"C:\\Daten\\OngoingProject\\BigIce\\Simulations\\Cooling\\TestCooling"} \\
\path{############} if runType = "runFromPath" \\
\textcolor{blue}{string} pathFolderToRun \path{"C:\\Daten\\OngoingProject\\BigIce\\Simulations\\GSHPCity-failed"} \\
\path{############} if runType = "runFromConfig" \\

\path{########} VARIATIONS \\

\textcolor{blue}{bool} combineAllCases True \# True for all combinations, False for one to one simulation ALL variation MUST have same lengh \\
\textcolor{blue}{variation} Ac AcollAp 1.5      \\
\textcolor{blue}{variation} Vice VIceS 0.3      \\

\path{######################} \\
\#FIXED NAMES CHANGED IN DECKS. \\
\path{######################} \\
\textcolor{blue}{deck} START 4344   \# START  is a variable that exist in one of the ddck files. \\
\textcolor{blue}{deck} STOP2 13104   \\
\textcolor{blue}{deck} maxIceFrac 0.8   \\ 
\textcolor{blue}{deck} ratioIceBegin 0   \\
\textcolor{blue}{deck} yBetweenHx string 0.05+0.05 \#used to add equations   \\
\textcolor{blue}{deck} TiceIni 25   \\

\path{######################} \\
\#CHANGE DDCK FILES \\
\path{######################} \\

\#\textcolor{blue}{changeDDckFile} nameToChange changedName1 changedName2 etc... \\
\textcolor{blue}{changeDDckFile} \path{CityKLO_dryN  CityDAV_dryN CityGVE_dryN} \\
\# we can use all path from the end if desired e.g. \path{SIA\Normal\CityKLO_dryN}\\
\#\textcolor{blue}{addDDckFolder} folderForChangeName1 folderForChangeName2 etc... \\
\textcolor{blue}{addDDckFolder} City1 City2 \# It will move results of changedName1 to City1 and changedName2 to City2 \\

\path{######################} \\
\#NAMES OF DDCK \\
\path{######################} \\
\path{HOME$ Generic\Head} \\
\path{HOME$  Control\Control-Cooling-888} \\
\path{HOME$  Control\Control-Cooling-863} \\
\path{HOME$  Control\Control-863_Printer} \\

\path{HOME$  DemandsDHW\DHW-MFH} \\
\path{HOME$  DemandsDHW\DHW-MFH_Printer} \\

\path{HOME$  weatherData\WeatherData} \\
\path{HOME$  weatherData\WeatherData_Printer} \\
\path{HOME$  weatherData\SIA\Normal\CityKLO_dryN} \\

\path{HOME$ Building\Bui-ISO-MFH2019-RadFloor-Cooling} \\
\path{HOME$ Building\Bui-ISO-MFH2019-RadFloor-Cooling_Printer} \\

\path{HOME$ IceStore\FP-Hx-IceStorage} \\
\path{HOME$ IceStore\IceStore-Cooling-861Generic} \\
\path{HOME$ IceStore\IceStore-861Generic_Printer} \\
\path{HOME$ IceStore\CellarIceCoupled} \\

\path{HOME$ SolarCollector\SolarCollector} \\
\path{HOME$ SolarCollector\SolarCollector_Printer} \\
\path{HOME$ SolarCollector\uncoveredNonSelectiveESSA} \\

\path{HOME$ HeatPump\HeatPumpBase} \\
\path{HOME$  HeatPump\BrineToWater\Cadena\ProDomo13-R410A\ProDomo13-R410A} \\
\path{HOME$  HeatPump\PrinterHP-SolarIce} \\

\path{HOME$  Tes\Tes-HxSolar-CircDirect} \\
\path{HOME$  Tes\TesPrinter-SolarIce} \\

\path{HOME$  HX\Const_Eff_HX} \\
\path{HOME$  Hydraulics\HydD_circOk} \\
\path{#HOME$ Hydraulics\pumps} \\

\path{HOME$  Printers\coolingPrinter} \\
\path{HOME$  Printers\MonthlyPrinter} \\
\path{HOME$  Printers\PrinterQvsT-SolarIce} \\
\path{HOME$  Printers\debugPrinter} \\

\path{HOME$  Generic\End} \\


\end{document}